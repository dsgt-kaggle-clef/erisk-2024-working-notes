\documentclass[]{style/ceurart}
\sloppy

\usepackage{listings}
\lstset{breaklines=true}

\begin{document}

\copyrightyear{2024}
\copyrightclause{Copyright for this paper by its authors.
  Use permitted under Creative Commons License Attribution 4.0
  International (CC BY 4.0).}

\conference{CLEF 2024: Conference and Labs of the Evaluation Forum, September 9-12, 2023, Grenoble, France}

\title{DS@GT eRisk 2024 Working Notes}

\author[1]{David Guecha}[
email=dahumada3@gatech.edu,
orcid=0009-0009-9855-5330
]
\author[1]{Aaryan Potdar}[
email=apotdar31@gatech.edu
]
\author[1]{Anthony Miyaguchi}[
orcid=0000-0002-9165-8718,
email=acmiyaguchi@gatech.edu,
]
\cormark[1]

\address[1]{Georgia Institute of Technology, North Ave NW, Atlanta, GA 30332}
\cortext[1]{Corresponding author.}


\begin{abstract}
The eRisk Challenge 2024 consists of 3 different tasks focused on the development of early risk prediction systems that use documents from social media to detect antisocial behaviors, signs of mental diseases or eating disorders, predictions are made using state-of-the-art natural language techniques on documents generated from social media users. We propose a system that predicts symptoms of depression selected from the Beck depression questionnaire BDI-II, using different architectures for the processing of sentences and doing a prediction based on multiclass classification, we propose a system to predict whether a user presents symptoms of eating disorders using a prediction system from similarities on vectors derived from sentence-transformers to the relevant symptoms. As a group we worked on two different tasks for the challenge, task 1 related to finding symptoms of depression from a questionnaire, and task 3 that deals with diagnosing eating disorders.

\end{abstract}

\begin{keywords}
  LaTeX class \sep
  paper template \sep
  paper formatting \sep
  CEUR-WS
\end{keywords}


\maketitle

\section{Introduction}

Introduction to the problem and the approach.

\section{Related Work}

Overview of related work.

\section{Task 1}

Task 1 description


\subsection{Task 1 Methodology}

A description of the data and methods.

For example, we use attention mechanisms \cite{vaswani2017attention} to improve the performance of a neural network on a classification task with multimodal data.

\subsection{Task 1 Results}

This section includes the results of the experiments, primarily in tabular or graphical form.

\subsection{Task 1 Discussion}

Discussion of the results and their implications.

\subsection{Task 1 Future Work}

What would you do next?


\section{Task 2}

Task 1 description


\subsection{Task 2 Methodology}

A description of the data and methods.

For example, we use attention mechanisms \cite{vaswani2017attention} to improve the performance of a neural network on a classification task with multimodal data.

\subsection{Task 2 Results}

This section includes the results of the experiments, primarily in tabular or graphical form.

\subsection{Task 2 Discussion}

Discussion of the results and their implications.

\subsection{Task 2 Future Work}

What would you do next?


\section{Conclusions}

Summary of the work and its contributions.

\section*{Acknowledgements}

Thank you to the DS@GT CLEF team for their support.

\bibliography{main}

% \appendix
% \section{Online Resources}

\end{document}